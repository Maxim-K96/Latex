%  1 #! python3
%  2 
%  3 import pyperclip
%  4 text = pyperclip.paste()
%  5 
%  6 lines = text.split('\n')
%  7 
%  8 for i in range(len(lines)):
%  9     if lines[i][3] == '-':
% 10         lines[i].replace('-', ' ')
% 11         lines[i] = '---\\textsl{ ' + lines[i].lstrip('   -') + '}' + ' \\\\'
% 12     else:
% 13         lines[i] = '\\hspace*{1.3em}' + lines[i].lstrip() + ' \\\\'
% 14 
% 15 text = '\n'.join(lines)
% 16 text.lstrip()
% 17 
% 18 pyperclip.copy(text)

%%%%%%%%%%%%%%%%%%%%%%%%%%%%%%%%%%%%%%%
% Настройки документа { Преамбула }   %
% %%%%%%%%%%%%%%%%%%%%%%%%%%%%%%%%%%%%%

% twoside - задает печать с разными полями на нечетных и четных страницах (как в книгах)
% oneside — печать с одинаковыми
% 
%%%%%%%%%%%%%%%%%%%%%%%%%%%%%%%%%%%%%%%
%
% a4paper 210 × 297(миллиметры)
% a5paper 148 × 210(миллиметры)
% b5paper 176 × 250(миллиметры)
% legalpaper 8.5 × 14(дюймы)
% executivepaper 7.25 × 10.5(дюймы)

%%%%%%%%%%%%%%%%%%%%%%%%%%%%%%%%%%%%%%%%

\documentclass[a4paper, 12pt, openany]{article}

%%%%%%%%%%%%%%%%%%%%%%%%%%%%%%%%%%%%%%%%%%%
%               виды документа            %
%%%%%%%%%%%%%%%%%%%%%%%%%%%%%%%%%%%%%%%%%%%

% article (статья)    report (отчёт)
% proc (доклад)       book (книга)
% letter (письмо)     slides (слайды)

%%%%%%%%%%%%%%%%%%%%%%%%%%%%%%%%%%%%%%%%%%%
%                 Стили                   %
%%%%%%%%%%%%%%%%%%%%%%%%%%%%%%%%%%%%%%%%%%%

% \pagestyle - задания стиля оформления страницы  
%	 эта команда имеет один обязательный аргумент — слово, обозначающее этот стиль:
% 		empty - нет ни колонтитулов, ни номеров страниц;
% 		plain - номера страниц ставятся внизу в середине строки,колонтитулов нет;
% 		headings - присутствуют колонтитулы (включающие в себя и номера страниц);
% 		myheadings - присутствуют колонтитулы, оформленные так же,как в предыдущем случае; отличие в том, что текст, печатающийся в колонтитулах 
%                   (в стандарт-ном случае это номера и названия разделов документа), 
%                    не порождаются LATEX’ом автоматически,а задается пользователем в явном виде.

%%%%%%%%%%%%%%

% \thispagestyle - задает стиль оформления одной отдельно взятой страницы. Те же аргументы что и у \pagestyle

%%%%%%%%%%%%%%%%%%%%%%%%%%%%%%%%%%%%%%%%%%%%

% \pagenumbering - чтобы страницы нумеровались не арабскими цифрами, что делается по умолчанию, 
%                               а римскими цифрами или буквами в алфавитном порядке.
%                  Аргументы:
%                             arabic - арабские цифры (1, 2, 3,. . . )
%                             roman  - римские цифры (i, ii, iii,. . . )
%                             Roman  - римские цифры (I, II, III,. . . )
%                             alph   - строчные буквы (a, b, c,. . . )
%                             Alph   - прописные буквы (A, B, C,. . . )
%
% Команда \pagenumbering не только меняет вид, в котором на печати представляются номера страниц, 
%   но и начинает счет страниц заново(это удобно, например, в тех случаях, когда страницы предисловия 
%   надо нумеровать римскими цифрами, а страницы основного текста заново нумеровать арабскими.

%%%%%%%%%%%%%%%%%%%%%%%%%%%%%%%%%%%%%%%%%%%%%%
%           Ширина текста                    %
%%%%%%%%%%%%%%%%%%%%%%%%%%%%%%%%%%%%%%%%%%%%%%

% \textwidth - Ширина текста на странице 
% 						если набор осуществляется в две колонки, 
%             то \textwidth включает в себя ширину обеих колонок и пробел между ними.
% Синтаксис:
%            \textwidth=7cm

% \oddsidemargin - величина левого поля при одностороннем наборе

%%%%%%%%%%%%%%%%%%%%%%%%%%%%%%%%%%%%%%%%%%%%%%
%              Пакеты                        %
%%%%%%%%%%%%%%%%%%%%%%%%%%%%%%%%%%%%%%%%%%%%%%

\usepackage[T2A]{fontenc}
\usepackage[utf8]{inputenc} % кодировка
\usepackage[english,russian]{babel} % поддержка русского и английского языка
\usepackage{indentfirst} % отступ в первом абзаце 
\usepackage{microtype}
\usepackage{misccorr}

% =================================================

% \usepackage{lscape}        % пакет для альбомного отображения 

% \begin{landscape}          % альбомный 

% \begin{wrapfigure}{i}{1\textwidth}
%     \begin{center}
%       \includegraphics[width=\textwidth]{< ... >}
%     \end{center}
% \end{wrapfigure}

% \end{landscape}

% ================================================

\usepackage{graphicx}                    % Картинки
\graphicspath{{image/}}
\DeclareGraphicsExtensions{.jpg,.png,.gif,.jpeg} 
\usepackage{wrapfig}                     % Обтекание картинок

% \begin{wrapfigure}{i}{0.3\textwidth}         % {o}; {i}; {r}
%     \begin{center}
%       \includegraphics[width=1\textwidth]{ < ... > }
%     \end{center}
%     \caption{ < Описание изображения > }
% \end{wrapfigure}

% =================================================

\usepackage{titlesec, blindtext, color}

% \titleformat{\section}[hang]{\normalfont\bfseries}{ \thesection. }{}{}  
% \titlespacing{\section}{\parindent}{4ex}{0pt}
% \renewcommand\thesubsection{\arabic{subsection}}

% \titleformat{\subsection}[hang]{\normalfont\bfseries}{ \thesection.\thesubsection}{0pt}{\Huge\bfseries} 

% ================================================

% \usepackage{dejavu}   % Шрифт
\usepackage{fancybox, fancyhdr}         % Колонтитулы 
\pagestyle{fancy}

% \fancyhead[R]{ ... }    % Верхний колонтитул справа
% \fancyhead[L]{ ... }    % Верхний колонтитул слева
% \thispagestyle{plain}   % Отсутствие верхних колонтитулов

% \thesection.\thesubsection  - номер_section.номер_subsection     


% ================================================

% \usepackage{geometry} % Меняем поля страницы

% \geometry{left=3.8cm}% левое поле
% \geometry{right=4.5cm}% правое поле
% \geometry{top=3cm}% верхнее поле
% \geometry{bottom=3cm}% нижнее поле

% ===============================================

% \usepackage{xcolor}

% \pagecolor[HTML]{FFFED6}   % Цвет фона документа

% ==============================================

% \renewcommand{\thechapter}{\Roman{chapter}} % Римские цифры chapter

% \parindent=2.5em

\begin{document}            % начало документа

%\autor{ ... } % автор документа

\title{}                    % заголовок
\date{\today}               % дата написания

\maketitle                 % печатает заголовок, список авторов и дату

\setcounter{tocdepth}{1}       % оторажение в содержании \path, \section
\tableofcontents{}

\newpage                   % Новая страница

%%%%%%%%%%%%%%%%%%%%%%%%%%%%%%%%%%%%%%%%%%%%%%
%               Шрифты                       %
%%%%%%%%%%%%%%%%%%%%%%%%%%%%%%%%%%%%%%%%%%%%%%
                                             
% \textup{Upright shape (прямое)}            %
% \textit{Italic shape (курсивное)}          %
% \textsl{Slanted shape (наклонное)}         %
% \textsc{Small caps shape (капитель)}       %
% \textmd{Medium series (средняя)}           %
% \textbf{Boldface series (полужирная)}      %
% \textrm{Roman family (романская)}          %
% \textsf{Sans serif family (рубленая)}      %
% \texttt{Typewriter family (машинописная)}  %

%%%%%%%%%%%%%%%%%%%%%%%%%%%%%%%%%%%%%%%%%%%%%%%

% Подстрочные примечания: \footnote           %

% Подчеркивание текста:   \underline          %

% Часть текста в рамкe:   \fbox               %

%%%%%%%%%%%%%%%%%%%%%%%%%%%%%%%%%%%%%%%%%%%%%%%
%                Размер текста                %
%%%%%%%%%%%%%%%%%%%%%%%%%%%%%%%%%%%%%%%%%%%%%%%

% \tiny           крошечный
% \scriptsize     индексный
% \footnotesize   подстрочный
% \small          маленький
% \normalsize     стандартный
% \large          большой
% \Large          Большой
% \LARGE          БОЛЬШОЙ
% \huge           огромный
% \Huge           Огромный

%%%%%%%%%%%%%%%%%%%%%%%%%%%%%%%%%%%%%%%%%%%%%%%%

%%%%%%%%%%%%%%%%%%%%%%%%%%%%%%%%%%%%%%%%%%%%%%%%
%             Маркированный список             %
%%%%%%%%%%%%%%%%%%%%%%%%%%%%%%%%%%%%%%%%%%%%%%%%

% \begin{itemize}
% \item < ... >
% \item < ... >
% \end{itemize}

%%%%%%%%%%%%%%%%%%%%%%%%%%%%%%%%%%%%%%%%%%%%%%%%

%%%%%%%%%%%%%%%%%%%%%%%%%%%%%%%%%%%%%%%%%%%%%%%%
%         Аннотация к тексту                   %
%%%%%%%%%%%%%%%%%%%%%%%%%%%%%%%%%%%%%%%%%%%%%%%%

% \begin{abstract}
% ...
% \end{abstract}

%%%%%%%%%%%%%%%%%%%%%%%%%%%%%%%%%%%%%%%%%%%%%%%%

%%%%%%%%%%%%%%%%%%%%%%%%%%%%%%%%%%%%%%%%%%%%%%%%
%                 Пробелы                      %
%%%%%%%%%%%%%%%%%%%%%%%%%%%%%%%%%%%%%%%%%%%%%%%%

% \vspace*{10mm}    %  Вертикальный отступ
% \hspace*{1.5em}   %  Горизонтальный отступ

%%%%%%%%%%%%%%%%%%%%%%%%%%%%%%%%%%%%%%%%%%%%%%%    

% ============================== CONTENT ========================================

% \begin{center}
% \section{ ... }
% \article{\large{\textbf{ ... }}}
% \end{center}

% \fancyhead[R]{ ... }    % Верхний колонтитул справа

% \vspace*{10mm}    %  Вертикальный отступ





\end{document}              % конец документа
